\documentclass{article}

\title{Slick Lang}
\date{}
\author{}

% \prooftree and related commands
\usepackage{ebproof}
% most symbols
\usepackage{amsmath}
\usepackage{amssymb}
\usepackage{amsthm}
\usepackage{stmaryrd}
\usepackage{graphicx}
% nice looking greek symbols
\usepackage{upgreek}
% \RHD
\usepackage[nointegrals]{wasysym}
\usepackage{xcolor}

\newcommand{\ignore}[1]{}
\newcommand{\todo}[1]{\colorbox{red}{#1}}
\newcommand{\consider}[1]{\colorbox{red}{!!!} #1 \colorbox{red}{!!!}}

\newcommand{\bnfvar}[1]{\langle \texttt{#1} \rangle}
\newcommand{\define}{::=}

\newcommand{\arrow}{\to}
\newcommand{\rcd}[1]{\{#1\}}
\newcommand{\vnt}[1]{\llbracket #1 \rrbracket}
\newcommand{\recursive}[2]{\upmu {#1}. {#2}}
\newcommand{\fa}[2]{\forall {#1}. {#2}}
\newcommand{\rowall}{\rotatebox[origin=c]{180}{\(\mathsf{R}\)}}
\newcommand{\far}[2]{\rowall {#1}. {#2}}
\newcommand{\rowtpvar}{\uprho}
\newcommand{\tpvar}{\upalpha}
\newcommand{\tpvaralt}{\upbeta}
\newcommand{\emptyrow}{\cdot}
\newcommand{\pat}{p}
\newcommand{\case}[2]{\texttt{case} \ {#1}: {#2}}

\newcommand{\spc}{\qquad}
\newcommand{\ctx}{\Upgamma}
\newcommand{\ctxalt}{\Updelta}
\newcommand{\ctxaltt}{\Uptheta}
\newcommand{\declCtx}{\Uppsi}
\newcommand{\derives}[3]{#1 \vdash #2 : #3}
\newcommand{\lbl}{\ell}
\newcommand{\expr}{e}
\newcommand{\var}{v}
\newcommand{\row}{r}
\newcommand{\rowm}{\uprho}
\newcommand{\tp}{A}
\newcommand{\tpalt}{B}
\newcommand{\tpaltt}{C}
\newcommand{\tpm}{\uptau}
\newcommand{\tpmalt}{\upsigma}
\newcommand{\marker}[1]{\RHD_{#1}}
\newcommand{\ev}{\hat}
\newcommand{\evar}[1][]{\ev \upalpha_{#1}}
\newcommand{\evaralt}{\ev \upbeta}
\newcommand{\tvar}{\alpha}
\newcommand{\ctxinout}[3]{#1 \vdash #2 \dashv #3}
\newcommand{\synth}[4]{\ctxinout {#1} {#2 \Rightarrow #3} {#4}}
\renewcommand{\check}[4]{\ctxinout {#1} {#2 \Leftarrow #3} {#4}}
\newcommand{\subsume}{<:}
\newcommand{\subsumes}[4]{\ctxinout {#1} {#2 \subsume #3} {#4}}
\newcommand{\synthesizes}{\Rightarrow \!\!\! \Rightarrow}
\newcommand{\presynth}[6]{\ctxinout {#1} {#2 #3 #4 \synthesizes #5} {#6}}
\newcommand{\app}{\bullet}
\newcommand{\appsynth}[5]{\presynth {#1} {#2} \app {#3} {#4} {#5}}

\newcommand{\B}{\mathcal{B}}
\newcommand{\C}{\mathcal{C}}
\newcommand{\D}{\mathcal{D}}
\newcommand{\rowvar}{\alpha_\rho}
\newcommand{\bottom}{\perp}
\newcommand{\prjSymbol}{.}
\newcommand{\prj}{\,\prjSymbol\,}
\newcommand{\instLSymbol}{\;\substack{<\\:=}\;}
\newcommand{\instRSymbol}{\;\substack{<\\=:}\;}
\newcommand{\apply}[1]{\left[#1\right]}

\newcommand{\rnil}{\{\}}
\newcommand{\rcons}[2]{\{#1 \,|\, #2\}}

\newcommand{\wf}[2]{#1 \vdash #2}

\newcommand{\subtype}{\le}
\newcommand{\subtypes}[3]{#1 \vdash #2 \le #3}

\newcommand{\declSynth}[3]{#1 \vdash #2 \Rightarrow #3}
\newcommand{\declCheck}[3]{#1 \vdash #2 \Leftarrow #3}
\newcommand{\declApSynth}[4]{#1 \vdash #2 \app #3 \synthesizes #4}
\newcommand{\declLookup}[4]{#1 \vdash #2 \# #3 \longrightarrow #4}

\newcommand{\instL}[4]{#1 \vdash #2 \instLSymbol #3 \dashv #4}
\newcommand{\instR}[4]{#1 \vdash #2 \instRSymbol #3 \dashv #4}

\newcommand{\lookup}[5]{#1 \vdash #2 \# #3 \longrightarrow #4 \dashv #5}

\newcommand{\extends}{\longrightarrow}

\newcommand{\deduct}[3][]
{
  \begin{prooftree}
    \hypo{#2}
    \infer1[\text{#1}]{#3}
  \end{prooftree}
}

\begin{document}

\maketitle

\section{Acknowledgements}

This type system is heavily adapted from the one presented in \textit{Complete
and Easy Bidirectional Typing for Higher-Rank Polymorphism} by Dunfield and
Krishnaswami.

Other references include \textit{Types and Programming Langauges} by Benjamin
Pierce and Chapter 10 of \textit{Advanced Types and Programming Languages},
``The Essence of ML Type Inference,'' by Fran\c{c}ois Pottier and Didier
R\'{e}my.

\section{Declarative Typing}

\subsection{Terms}

\[
  \begin{array}{llll}
    \text{labels (includes capital labels)} & \lbl & & \\
    \text{capital labels} & L & & \\
    \text{patterns} & \pat & & \\
    \text{variables} & \var & & \\
    \text{expressions} & \expr & \define & \\
    & \text{functions} & | & \backslash \var
                           \arrow \expr \\
    & \text{applications} & | & \expr_1 \expr_2 \\
    & \text{sequences} & | & \expr_1 ; \expr_2 \\
    & \text{assignments} & | & \var := \expr \\
    & \text{annotations} & | & \expr : \tp \\
    & \text{records} & | & \rcd{\lbl_1=\expr_1, \lbl_2=\expr_2,\dots} \\
    & \text{variants} & | & L \ \expr \\
    & \text{case} & | & \case{\expr}{| \pat_1 \arrow \expr_1 | \pat_2 \arrow
                        \expr_2 \dots} \\
  \end{array}
\]

\subsection{Types and type machinery}

\[
  \begin{array}{lllrl}
    \text{labels} & \lbl & & \\
    \text{type variables} & \tpvar, \tpvaralt & & \\
    \text{row type variables} & \rowtpvar & & \\
    \text{contexts} & \declCtx & \define & \cdot \\
    & & | & v = \tp, \declCtx \\
    \text{monomorphic types} & \tpm, \tpmalt & \define & \rcd \uprho  \\
    & & | & \vnt \uprho \\
    & & | & \recursive \tpvar \tpm \\
    & & | & \tpm \arrow \tpmalt \\
    \text{types} & \tp, \tpalt & \define & \fa \tpvar \tp \\
    & & | & \tpm \\
    \text{rows} & \row & \define & \emptyrow  \\
    & & | & \lbl : \tp, \row \\
    \text{monomorphic rows} & \rowm & \define & \emptyrow  \\
    & & | & \lbl : \tpm, \rowm  \\
  \end{array}
\]

\subsection{Typing rules}

This section is a little incomplete since the Algorithmic Typing is what matters
more for the implementation.

\subsubsection{Functions}
\[
  \deduct
  [(Var)]
  {\declCtx, \var = \tp, \declCtx' \spc \var \notin \declCtx}
  {\derives {\declCtx, \var = \tp, \declCtx'} \var \tp}
  \spc
  \deduct
  [(Function)]
  {\derives {\var = \tp, \declCtx} \expr \tpalt}
  {\derives \declCtx {\backslash \var \arrow \expr} { \tp
      \arrow \tpalt}}
\]
\[
  \deduct
  [(App)]
  {\derives \declCtx {\expr_1} {\tp \arrow \tpalt} \spc \derives \declCtx
    {\expr_2} {\tp}}
  {\derives \declCtx {\expr_1 \expr_2} \tpalt}
\]

\subsubsection{Records}
\[
  \deduct
  [(Empty record)]
  {}
  {\derives \declCtx {\rcd{}} \emptyrow}
  \spc
  \deduct
  [(Record)]
  {\derives \declCtx {\rcd {rcd}} {\rcd \row}}
  {\derives \declCtx {\rcd{\lbl=e,rcd}} {\rcd{\lbl : \tp, \row}}}
\]

\section{Algorithmic typing}

\subsection{Terms}

Same as those from the Declarative typing section.

\subsection{Types and type machinery}

\[
  \begin{array}{lllrl}
    \text{labels} & \lbl & & \\
    \text{existential variables} & \evar & & \\
    \text{existential row variables} & \rowvar & & \\
    \text{type variables} & \tpvar, \tpvaralt & & \\
    \text{row type variables} & \rowtpvar & & \\
    \text{base types} & \B & \define & \texttt{Int} | \texttt{Bool} | \dots\\
    \text{contexts} & \ctx, \ctxalt, \ctxaltt & \define & \cdot \\
    & \text{vars} & | & \var : \tp, \ctx \\
    & \text{solved evars} & | & \evar : \tp, \ctx \\
    & \text{solved row evars} & | & \rowvar : \row, \ctx \\
    & \text{evars} & | & \evar, \ctx \\
    & \text{type vars} & | & \tvar, \ctx \\
    & \text{markers} & | & \marker \evar, \ctx \\
    \text{monomorphic types} & \tpm, \tpmalt & \define & \rcd \uprho  \\
    & & | & \vnt \uprho \\
    & & | & \recursive \tpvar \tpm \\
    & & | & \tpm \arrow \tpmalt \\
    & & | & \B \\
    \text{types} & \tp, \tpalt & \define & \fa \tpvar \tp \\
    & & | & \tpm \\
    \text{rows} & \row & \define & \emptyrow  \\
    & & | & \lbl : \tp, \row \\
    \text{monomorphic rows} & \rowm & \define & \emptyrow  \\
    & & | & \lbl : \tpm, \rowm  \\
  \end{array}
\]

\subsection{A note on quantifiers}

You'll note that there are two quantifiers in Slick: \(\fa \tpvar A\) and \(\far
\rowvar A\). This is because row tails don't strictly contain types, so it makes
sense to have a quantifier over just the universe of rows, which is what the
latter is.

In practice, the two are checked equivalently. All rules pertaining to \(\fa
\tpvar A\), unless stated otherwise, have mirrored counterparts for \(\far
\rowvar A\).

\subsection{Typing rules}
We define algorithmic typing with the following judgments:
\[
\check{\ctx}{e}{A}{\ctxalt}
\spc
\synth{\ctx}{e}{A}{\ctxalt}
\]
which respectively represent type checking (inputs: $\ctx$, $\expr$, $A$; output: $\ctxalt$) and type synthesis (inputs: $\ctx$, $\expr$; outputs: $A$, $\Delta$).

We also define a binary algorithmic judgement:

\[
\presynth{\ctx}{X}{\square}{Y}{Z}{\ctxalt}
\]

which represents a binary judgement \(\square\) under the context \(\ctx\) on
values \(X\) and \(Y\) that synthesizes \(Z\) with output context \(\ctxalt\).
For example, the syntax that Dunfield and Krishnaswami use for function
application synthesis judgements would be

\[
\appsynth \ctx A e C \ctxalt
\]

which means that under context \(\ctx\), \(A\) applied to the term \(e\)
synthesizes output type \(C\) and context \(\ctxalt\).

\[
  \deduct[(Var)]
  {
    (x : A) \in \ctx
  }
  { \synth{\ctx}{x}{A}{\ctx} }
  \spc
  \deduct[(Sub)]
  {
    \synth{\ctx}{e}{A}{\ctxaltt} \spc
    \subsumes{\ctxaltt}{\apply\ctxaltt A}{\apply\ctxaltt B}{\ctxalt}
  }
  { \check{\ctx}{e}{B}{\ctxalt} }
\]

\[
  \deduct[(Annotation)]
  { \ctx \vdash A \spc \check{\ctx}{e}{A}{\ctxalt} }
  { \synth{\ctx}{(e : A)}{A}{\ctxalt} }
  \spc
  \deduct[(\(\forall\) I)]
  { \check{\ctx, \alpha}{e}{A}{\ctxalt, \alpha, \ctxaltt} }
  { \check{\ctx}{e}{\forall \alpha. A}{\ctxalt} }
\]

\[
  \deduct[(\(\to\)I)]
  { \check{\ctx, \var : \tp}{e}{\tpalt}{\ctxalt, \var : \tp, \ctxaltt} }
  { \check{\ctx}{\backslash \var \arrow e}{ \tp \arrow \tpalt}{\ctxalt} }
\]

\[
  \deduct[(\(\to\)I\(\implies\))] { \check{\ctx, \marker\evar, \evar, \evaralt,
      \var : \evar}{e}{\evaralt}{\ctxalt, \marker\evar, \ctxaltt} \spc \tau =
    \apply \ctxalt (\evar \arrow \evaralt) \spc \overline\evar =
    \text{unsolved}(\tau)} { \synth{\ctx}{\backslash \var \arrow e}{{\forall
        \overline{\upalpha}. \tpm[\overline \evar \mapsto \overline
        \upalpha]}}{\ctxalt} }
\]

\[
  \deduct[(\(\to\)E)]
  { \synth{\ctx}{e_1}{A}{\ctxaltt} \spc \presynth{\ctxaltt}{\apply\ctxaltt A}{\app}{e_2}{C}{\ctxalt} }
  { \synth{\ctx}{e_1 \ e_2}{C}{\ctxalt}  }
  \spc
  \deduct[(\(\forall\) App)]
  { \presynth{\ctx, \ev\alpha}{A[\alpha := \ev\alpha]}{\app}{e}{C}{\ctxalt} }
  { \presynth{\ctx}{\forall \alpha. A}{\app}{e}{C}{\ctxalt} }
\]

\[
  \deduct[(\(\to\)App)]
    { \check{\ctx}{e}{A}{\ctxalt} }
    { \presynth{\ctx}{A \to C}{\app}{e}{C}{\ctxalt} }
  \spc
  \deduct[(\(\upmu\)App)]
    { \presynth{\ctx}{\tp[\tpvar \mapsto \recursive \tpvar \tp]}{\app}{e}{C}{\ctxalt} }
    { \presynth{\ctx}{\recursive \tpvar \tp}{\app}{e}{C}{\ctxalt} }
\]


\[
  \deduct[(\(\ev\alpha\)App)]
    {
      \check{\ctx[\ev{\alpha_2}, \ev{\alpha_1}, \ev \alpha = \ev{\alpha_1} \to
        \ev{\alpha_2}]}{e}{\ev{\alpha_1}}{\ctxalt}
    }
    { \presynth{\ctx[\ev\alpha]}{\ev\alpha}{\app}{e}{\ev\alpha_2}{\ctxalt} }
\]


\[
\deduct[(\{\}I)]{}{\check{\ctx}{\{\}}{\{\}}{\ctx}}
\spc
\deduct[(\{\}I\(\implies\))]{}{\synth{\ctx}{\{\}}{\{\}}{\ctx}}
\]

\[
  \deduct[(RcdI)]
  { \check{\ctx}{e}{A}{\ctxaltt} \spc \check{\ctxaltt}{\apply\ctxaltt \rcd {rcd}}{\apply\ctxaltt\row}{\ctxalt} }
  { \check{\ctx}{\{\lbl : e , rcd \}}{\{\lbl : A, \row\}}{\ctxalt} }
  \spc
  \deduct[(RcdI\(\implies\))]
  { \synth{\ctx}{e}{A}{\ctxaltt} \spc \synth{\ctxaltt}{\apply\ctxaltt \rcd
        {rcd}}{\apply\ctxaltt \row}{\ctxalt} }
  { \synth{\ctx}{\{\lbl : e , rcd \}}{\{\lbl : A, \row\}}{\ctxalt} }
\]

\[
  \deduct[(Prj)]
  { \synth{\Gamma}{e}{A}{\Theta} \spc
    \presynth{\Theta}{\apply\Theta A}{\prj}{\lbl}{C}{\Delta}
  }
  { \synth{\Gamma}{e\#\lbl}{C}{\Delta} }
  \spc
  \deduct[(\(\forall\) Prj)]
  { \presynth{\Gamma, \ev\alpha}{A[\alpha := \ev\alpha]}{\prj}{\lbl}{C}{\Delta} }
  { \presynth{\Gamma}{\forall \alpha. A}{\prj}{\lbl}{C}{\Delta} }
\]

\[
  \deduct[(RcdPrjR)]
    { \lookup{\Gamma}{R}{l}{C}{\Delta} }
    { \presynth{\Gamma}{R}{\prj}{\lbl}{C}{\Delta} }
  \spc
  \deduct[(\(\upmu\)Prj)]
    { \presynth{\ctx}{\tp[\tpvar \mapsto \recursive \tpvar \tp]}{\prj}{\lbl}{C}{\ctxalt} }
    { \presynth{\ctx}{\recursive \tpvar \tp}{\prj}{\lbl}{C}{\ctxalt} }
\]

\noindent
We define record lookup $\lookup{\Gamma}{\rho}{\lbl}{A}{\Delta}$ as follows (inputs: $\Gamma$, $\rho$, $l$; outputs: $A$, $\Delta$):
\[
\deduct[(lookupYes)]{}{\lookup{\Gamma}{\{\lbl : A, R\}}{\lbl}{A}{\Gamma}}
\spc
\deduct[(lookupNo)]
  {\lbl \neq \lbl' \spc \lookup{\Gamma}{\{R\}}{\lbl}{A}{\Delta}}
  {\lookup{\Gamma}{\{\lbl' : A', R\}}{\lbl}{A}{\Delta}}
\]

\[
\deduct[(Lookup \(\ev\alpha\))]
  { }
  { \lookup
      {\Gamma[\ev\alpha]}
      {\ev\alpha}
      {\lbl}
      {\ev\alpha_0}
      {\Gamma[\ev\alpha_0, \ev\alpha_1, \ev\alpha = \{\lbl : \ev\alpha_0, \ev\alpha_1\}] }
  }
\]

\[
\deduct[(Lookup RowTail)]
  { }
  { \lookup
      {\Gamma[\ev\alpha]}
      {\{\ev\alpha\}}
      {\lbl}
      {\ev\alpha_0}
      {\Gamma[\ev\alpha_0, \ev\alpha_1, \ev\alpha = (\lbl : \ev\alpha_0, \ev\alpha_1)] }
  }
\]

\subsection{Subsumption}
We define the algorithmic subsumption:
\[
\subsumes{\Gamma}{A_0}{A_1}{\Delta}
\]
which represents $A_0$ subsumes $A_1$ with input context $\Gamma$ and output
context $\Delta$. Subsumption is like subtyping, but only applies to
quantifiers. Everything else must be strict equality (for now, this also means
records, so you can't use \(\{\lbl_1: \texttt{Bool}, \lbl_2: \texttt{Num}\}\) in
place of \(\{\lbl_1 : \texttt{Bool}\}\) even though you really \emph{should} be
able to).
\[
  \deduct[(EVar)]{}{\subsumes{\ctx[\ev\alpha]}{\ev\alpha}{\ev\alpha}{\ctx[\ev\alpha]}}
  \spc
  \deduct[(Var)]{}{\subsumes{\ctx[\alpha]}{\alpha}{\alpha}{\ctx[\alpha]}}
  \spc
  \deduct[(Const)]{}{\subsumes{\ctx}{\B}{\B}{\ctx}}
\]

\[
  \deduct[(\(\forall\)L)]
  { \subsumes{\ctx, \marker{\ev\alpha}, \ev\alpha}{A[\alpha := \ev\alpha]}{B}{\Delta, \marker{\ev\alpha}, \Theta} }
  { \subsumes{\ctx}{\forall \alpha. A}{B}{\Delta} }
  \spc
  \deduct[(\(\forall\)R)]
  { \subsumes{\ctx, \alpha}{A}{B}{\Delta, \alpha, \Theta} }
  { \subsumes{\ctx}{A}{\forall \alpha. B}{\Delta} }
\]

\[
  \deduct[(\(\to\))]
  { \subsumes{\ctx}{B_1}{A_1}{\Theta} \spc \subsumes{\Theta}{\apply\Theta A_2}{\apply\Theta B_2}{\Delta} }
  { \subsumes{\ctx}{A_1 \to A_2}{B_1 \to B_2}{\Delta} }
\]

\[
  \deduct[InstantiateL]
  {\evar \notin FV(A) \spc \instL{\ctx[\evar]}{\ev
      \alpha}{A}{\Delta}}
  {\subsumes{\ctx[\evar]}{\evar}{A}{\Delta}}
  \spc
  \deduct[InstantiateR]
  {\evar \notin FV(A) \spc \instR{\ctx[\evar]}{A}{\ev
      \alpha}{\Delta}}
  {\subsumes{\ctx[\evar]}{A}{\evar}{\Delta}}
\]

\[
  \deduct[InstantiateL\(\upmu\)]
  {\evar \in FV(A) \spc \instL{\ctx[\evar]}{\ev
      \alpha}{\recursive \tpvar A[\evar \mapsto \tpvar]}{\Delta}}
  {\subsumes{\ctx[\evar]}{\evar}{A}{\Delta}}
\]
\[
  \deduct[InstantiateR\(\upmu\)]
  {\evar \in FV(A) \spc \instR{\ctx[\evar]}{\recursive \tpvar A[\evar \mapsto \tpvar]}{\ev
      \alpha}{\Delta}}
  {\subsumes{\ctx[\evar]}{A}{\evar}{\Delta}}
\]

The above rules for recursive types will likely be changed to be more
restrictive: probably only when \(A\) is a record or variant (or perhaps
contains one).

\[
  \deduct[(Record)]{\subsumes{\ctx}{R_0}{R_1}{\Delta}}{\subsumes{\ctx}{\{R_0\}}{\{R_1\}}{\Delta}}
  \spc
  \deduct[(Variant)]{\subsumes{\ctx}{R_0}{R_1}{\Delta}}{\subsumes{\ctx}{\vnt {R_0}}{\vnt
    {R_1}}{\Delta}}
\]

For this rule, we treat the rows as sets and assume they are reordered so that
the matching labels are at the front of the row. An algorithmic implementation
would want to deal with the recursive and base cases by looking at the set
intersection and difference of the rows.

\[
  \deduct[(Row)]
  {\subsumes{\Gamma}{A}{B}{\Theta}
    \spc
    \subsumes{\Theta}{[\Theta] R_1}{[\Theta] R_2}{\Delta}
  }
  { \subsumes{\Gamma}{\lbl : A, R_1}{\lbl : B, R_2}{\Delta} }
  \spc
  \deduct[(Row Nil)]{}{ \subsumes{\Gamma}{\cdot}{\cdot}{\Gamma} }
\]

\[
  \deduct[(RowMissingL)]
  {\instL{\Gamma[\evar]}{\evar}{r}{\Delta}}
  {\subsumes{\Gamma[\evar], \cdot}{\evar}{r}{\Delta}}
  \spc
  \deduct[(RowMissingR)]
  {\instR{\Gamma[\evar]}{r}{\evar}{\Delta}}
  {\subsumes{\Gamma[\evar], \cdot}{r}{\evar}{\Delta}}
\]

This rule also treats the rows as sets and assumes that there are no equal
labels between the two rows. Assume an analagous rule for a different order of
EVars (the order doesn't matter).
\[
  \deduct[(RowMissingLR)]
  {\subsumes{\Gamma}{\evar}{m_1 : B_1, m_2 : B_2, \dots}{\Theta} \spc
    \subsumes{\Theta}{\lbl_1 : \apply\Theta A_1, \lbl_2 : \apply\Theta A_2,
      \dots}{\evaralt}{\Delta}}
  {\subsumes{\Gamma[\evar][\evaralt]}{\lbl_1 : A_1, \lbl_2 : A_2, \dots, \evar}{m_1 :
      B_1, m_2 : B_2, \dots, \evaralt}{\Delta}}
\]

\[
  \deduct[(RowTailReach)]
  {}
  {\subsumes{\Gamma[\evar][\evaralt]}{\evar}{\evaralt}{\Gamma[\evar][\evaralt=\evar]}}
\]


\subsection{Instantiation}

Instantiation is a judgement that solves an EVar, either on the left or right
side of the judgement. It is important to recurisvely assign ``helper'' EVars to
match the shape of whatever the EVar is being instantiated to instead of blindly
assigning it, as there may be EVars inside of the type it is being assigned to.

The EVar is instantiated so that it subsumes or is subsumed by the type,
depending on the type of instantiation (left or right, respectively).

\consider{Need an instantiation for rows that is like InstRcd}

\subsubsection{Left Instantiation}

\[
  \deduct[InstLSolve]
  {\Gamma \vdash \B}
  {\instL{\Gamma[\ev\alpha]}{\ev \alpha}{\B}{\Gamma[\ev\alpha = \B]}}
  \spc
  \deduct[InstLReach]
  {}
  {\instL{\Gamma[\ev\alpha][\ev\beta]}{\ev \alpha}{\ev
      \beta}{\Gamma[\ev\alpha][\ev\beta = \ev\alpha]}}
\]

\[
  \deduct[InstLArr] {\instR{\Gamma[\evar[2], \evar[1], \evar = \evar[1] \to
      \evar[2]]}{A_1}{\evar[1]}{\Theta} \spc \instL{\Theta}{\evar[2]}{\apply \Theta
      A_2}{\Delta} } {\instL{\Gamma[\evar]}{\evar}{A_1 \to A_2}{\Delta}}
\]

\[
  \deduct[InstLRcd]
  {
    \begin{array}{l}
     \instL{\Gamma_0[(\evar[k+1]), \evar[k], \dots, \evar[1], \evar=\{\lbl_1 : \evar[1],
      \lbl_2 : \evar[2], \dots, \lbl_k : \evar[k], (\evar[k+1])\}]}{\evar[1]}{A_1}{\Gamma_1} \\
    \instL{\Gamma_1}{\evar[2]}{\apply{\Gamma_1} A_2}{\Gamma_2} \spc \cdots \spc
     (\instL{\Gamma_k}{\evar[k+1]}{\apply{\Gamma_k}\evaralt}{\Delta})
  \end{array}
  }
  {\instL{\Gamma_0[\evar]}{\evar}{\{\lbl_1:A_1, \lbl_2 : A_2, \dots, \lbl_{k-1} : A_{k-1}, (\evaralt)\}}{\Delta}}
\]
In the above rule, the parentheticals only come into play if there is a row tail
in the record. If there isn't, assume that \(\Delta = \Gamma_k\).

\[
  \deduct[InstLAllR]
  { \instL{\Gamma[\evar], \beta}{\evar}{B}{\Delta, \beta, \Delta'} }
  { \instL{\Gamma[\evar]}{\evar}{\forall \beta. B}{\Delta} }
\]

\subsubsection{Right Instantiation}

\[
  \deduct[InstRSolve]
  {\Gamma \vdash \B}
  {\instR{\Gamma[\ev\alpha]}{\B}{\ev \alpha}{\Gamma[\ev\alpha = \B]}}
  \spc
  \deduct[InstRReach]
  {}
  {\instR{\Gamma[\ev\alpha][\ev\beta]}{\ev
      \beta}{\evar}{\Gamma[\ev\alpha][\ev\beta = \ev\alpha]}}
\]

\[
  \deduct[InstRArr] {\instL{\Gamma[\evar[2], \evar[1], \evar = \evar[1] \to
      \evar[2]]}{\evar[1]}{A_1}{\Theta} \spc \instR{\Theta}{\apply \Theta
      A_2}{\evar[2]}{\Delta} } {\instR{\Gamma[\evar]}{A_1 \to A_2}{\evar}{\Delta}}
\]

\[
  \deduct[InstRRcd]
  {
    \begin{array}{l}
     \instR{\Gamma_0[(\evar[k+1]), \evar[k], \dots, \evar[1], \evar=\{\lbl_1 : \evar[1],
      \lbl_2 : \evar[2], \dots, \lbl_k : \evar[k], (\evar[k+1])\}]}{A_1}{\evar[1]}{\Gamma_1} \\
    \instR{\Gamma_1}{\apply{\Gamma_2} A_2}{\evar[2]}{\Gamma_3} \spc \cdots \spc
     (\instR{\Gamma_k}{\evaralt}{\evar[k+1]}{\Delta})
  \end{array}
  }
  {\instR{\Gamma_0[\evar]}{\{\lbl_1:A_1, \lbl_2 : A_2, \dots, \lbl_{k-1} : A_{k-1}, (\evaralt)\}}{\evar}{\Delta}}
\]
In the above rule, the parentheticals only come into play if there is a row tail
in the record. If there isn't, assume that \(\Delta = \Gamma_k\).

\[
  \deduct[InstRAllL]
  { \subsumes{\Gamma[\evar], \marker{\ev\alpha}, \evaralt}{A[\beta := \evaralt]}{\evar}{\Delta, \marker{\evaralt}, \Delta'} }
  { \subsumes{\Gamma[\evar]}{\forall \beta. B}{\evar}{\Delta} }
\]


\end{document}
